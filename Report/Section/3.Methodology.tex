\section{Methodology}

\subsection{Faithfulness Evaluation Framework Overview}
The faithfulness evaluation framework is designed to provide a comprehensive assessment of LLM outputs through multiple dimensions. At its core, the framework employs advanced natural language processing techniques, including pre-trained language models for semantic analysis and sophisticated metrics for various aspects of output evaluation.

\subsubsection{Framework Definition}
The framework focuses on evaluating the faithfulness of LLM outputs by:
\begin{itemize}
    \item Assessing multiple dimensions of response quality
    \item Employing sophisticated NLP techniques
    \item Utilizing pre-trained language models
    \item Implementing dynamic weight adjustment
\end{itemize}

\subsubsection{Evaluation Process}
The evaluation workflow consists of four main stages:
\begin{enumerate}
    \item Input processing and tokenization using specialized NLP tools
    \item Multi-dimensional metric calculation for each response
    \item Dynamic weight adjustment based on initial results
    \item Generation of comprehensive evaluation reports and visualizations
\end{enumerate}

\subsection{Evaluation Metrics}
Our framework implements six core metrics, each designed to capture different aspects of output faithfulness:

\subsubsection{Factual Accuracy}
Evaluates the accuracy of response content against reference material:
\begin{itemize}
    \item \textbf{Semantic Similarity}: Using sentence-transformers for embedding-based comparison
    \item \textbf{Key Fact Matching}: Identifying and verifying critical factual elements
    \item \textbf{Numerical Accuracy}: Specific verification of numerical values and statistics
\end{itemize}

\subsubsection{Logical Coherence}
Assesses the internal logical structure and flow:
\begin{itemize}
    \item \textbf{Inter-sentence Coherence}: Analyzing semantic relationships between sentences
    \item \textbf{Argumentation Structure}: Evaluating logical flow and reasoning patterns
    \item \textbf{Logical Connector Usage}: Examining transition words and phrases
\end{itemize}

\subsubsection{Context Relevance}
Measures alignment with provided context:
\begin{itemize}
    \item \textbf{Semantic Relevance}: Computing contextual similarity scores
    \item \textbf{Key Information Coverage}: Assessing critical context elements
    \item \textbf{Topic Consistency}: Evaluating topic adherence
\end{itemize}

\subsubsection{Interpretative Reasoning}
Analyzes reasoning quality and interpretation:
\begin{itemize}
    \item \textbf{Reasoning Patterns}: Detecting and evaluating reasoning structures
    \item \textbf{Process Completeness}: Assessing reasoning chain completeness
    \item \textbf{Context-based Conclusions}: Validating contextual inferences
\end{itemize}

\subsubsection{Information Completeness}
Evaluates response comprehensiveness:
\begin{itemize}
    \item \textbf{Keyword Coverage}: Analyzing essential keyword presence
    \item \textbf{Information Depth}: Measuring content depth
    \item \textbf{Response Comprehensiveness}: Assessing overall completeness
\end{itemize}

\subsubsection{Hallucination Score}
Identifies and quantifies potential hallucinations:
\begin{itemize}
    \item \textbf{Context Alignment}: Measuring contextual consistency
    \item \textbf{Fact Verification}: Identifying unsupported claims
    \item \textbf{Source Tracing}: Validating information sources
\end{itemize}

\subsection{Dynamic Weight Adjustment}
The framework implements an adaptive weight adjustment mechanism that modifies metric weights based on evaluation scenarios and response characteristics.

\subsubsection{Weight Adjustment Mechanism}
The base weight distribution is as follows:
\begin{itemize}
    \item Factual Accuracy: 25\%
    \item Logical Coherence: 15\%
    \item Context Relevance: 15\%
    \item Interpretative Reasoning: 15\%
    \item Information Completeness: 15\%
    \item Hallucination Score: 15\%
\end{itemize}

\subsubsection{Adjustment Rules}
Weights are dynamically adjusted based on specific triggers:

\begin{itemize}
    \item \textbf{Low Factual Accuracy Scenario} (accuracy < 0.5):
    \begin{itemize}
        \item Factual Accuracy weight: 35\%
        \item Hallucination Score weight: 20\%
        \item Other metrics: 45\% (equally distributed)
    \end{itemize}
    
    \item \textbf{High Hallucination Scenario} (hallucination score < 0.5):
    \begin{itemize}
        \item Hallucination Score weight: 25\%
        \item Factual Accuracy weight: 30\%
        \item Other metrics: 45\% (equally distributed)
    \end{itemize}
\end{itemize}

This dynamic adjustment ensures that the evaluation framework adapts to specific challenges in different response scenarios, providing more accurate and relevant assessments of LLM output faithfulness.

\subsection{Implementation Details}
The framework implementation consists of three core components:

\subsubsection{Evaluation Core (eval.py)}
The \texttt{FaithfulnessEval} class implements the core evaluation process:
\begin{itemize}
    \item \textbf{Core Evaluation Process}:
    \begin{itemize}
        \item Sample loading and preprocessing
        \item Model response acquisition
        \item Metric calculation and scoring
        \item Overall faithfulness assessment
    \end{itemize}
    \item \textbf{Key Features}:
    \begin{itemize}
        \item Robust error handling
        \item Configurable evaluation parameters
        \item Progress tracking and logging
        \item Result aggregation
    \end{itemize}
\end{itemize}

\subsubsection{Metrics Implementation (metrics.py)}
The \texttt{FaithfulnessMetrics} class provides:
\begin{itemize}
    \item \textbf{Metric Calculations}:
    \begin{itemize}
        \item Implementation of all six core metrics
        \item Pre-trained model integration for semantic analysis
        \item NLTK resource initialization
        \item Text processing utilities
    \end{itemize}
    \item \textbf{Features}:
    \begin{itemize}
        \item Configurable metric parameters
        \item Caching for efficiency
        \item Extensible metric framework
    \end{itemize}
\end{itemize}

\subsubsection{Report Generation (report.py)}
The \texttt{FaithfulnessReport} class handles:
\begin{itemize}
    \item \textbf{Report Generation}:
    \begin{itemize}
        \item Comprehensive evaluation reports
        \item Visualization integration
        \item Raw data export
    \end{itemize}
    \item \textbf{Visualization Types}:
    \begin{itemize}
        \item Performance radar charts
        \item Correlation heatmaps
        \item Score distributions
        \item Trend analysis plots
    \end{itemize}
\end{itemize}

\subsection{Sample Type-Specific Weights}
Different types of evaluation samples have specific metric weight distributions:

\begin{table}[h]
\centering
\caption{Metric Weights by Sample Type}
\label{tab:metric_weights}
\begin{tabular}{|l|c|c|c|c|c|c|}
\hline
\textbf{Type} & \textbf{FA} & \textbf{LC} & \textbf{CR} & \textbf{IR} & \textbf{IC} & \textbf{HS} \\
\hline
General & 0.30 & 0.20 & 0.15 & 0.15 & 0.10 & 0.10 \\
Medical & 0.35 & 0.15 & 0.15 & 0.15 & 0.10 & 0.10 \\
Scientific & 0.35 & 0.20 & 0.10 & 0.15 & 0.10 & 0.10 \\
Historical & 0.30 & 0.20 & 0.15 & 0.15 & 0.10 & 0.10 \\
Legal & 0.30 & 0.25 & 0.15 & 0.15 & 0.10 & 0.05 \\
\hline
\end{tabular}
\end{table}

\subsection{Technical Implementation}
The framework utilizes advanced NLP techniques and models:

\subsubsection{Text Embedding}
Sentence-Transformers model is used for text embedding:
\begin{verbatim}
def get_embeddings(self, text: str) -> np.ndarray:
    inputs = self.tokenizer(
        text, 
        return_tensors="pt", 
        padding=True, 
        truncation=True
    )
    outputs = self.model(**inputs)
    embeddings = outputs.last_hidden_state.mean(dim=1)
    return embeddings.detach().numpy()
\end{verbatim}

\subsubsection{Evaluation Process Flow}
The evaluation process follows a systematic workflow:
\begin{enumerate}
    \item \textbf{Model Initialization}:
    \begin{itemize}
        \item Configuration of API parameters and rate limits
        \item Environment setup and dependency verification
        \item Loading and validation of evaluation samples
        \item Initialization of metric calculation components
    \end{itemize}

    \item \textbf{Evaluation Execution}:
    \begin{itemize}
        \item Sequential processing of evaluation samples
        \item Real-time calculation of faithfulness metrics
        \item Dynamic adjustment of metric weights
        \item Continuous monitoring of evaluation progress
    \end{itemize}

    \item \textbf{Results Collection}:
    \begin{itemize}
        \item Aggregation of individual sample results
        \item Computation of type-specific performance metrics
        \item Generation of overall evaluation metrics
        \item Validation of collected results
    \end{itemize}
\end{enumerate}

\subsubsection{Logging and Results Storage}
The framework implements comprehensive logging and storage:
\begin{itemize}
    \item \textbf{Logging System}:
    \begin{itemize}
        \item Detailed timestamp-based execution logs
        \item Real-time progress tracking
        \item Error handling and reporting
        \item Performance monitoring
    \end{itemize}

    \item \textbf{Results Storage}:
    \begin{itemize}
        \item Structured JSON format
        \item Version control for reproducibility
        \item Automated backup mechanisms
        \item Data integrity validation
    \end{itemize}
\end{itemize}
