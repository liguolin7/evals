\section{Implementation}

\subsection{Project Structure}
The project follows a well-organized modular architecture:

\begin{lstlisting}[
    language=Text,
    breaklines=true,
    basicstyle=\ttfamily\footnotesize,
    commentstyle=\color{gray}
]
project_root/
|-- evals/
|   `-- elsuite/
|       `-- faithfulness/
|           |-- __init__.py
|           |-- eval.py           # Core evaluation implementation
|           |-- metrics.py        # Metric calculation functions
|           |-- report.py         # Report generation
|           |-- run.py           # Command line interface
|           `-- utils.py         # Utility functions
|-- scripts/
|   `-- generate_visualization.py  # Visualization tools
|-- logs/                         # Evaluation logs
|   `-- faithfulness_eval_*.log   # Timestamped evaluation logs
|-- results/                      # Detailed evaluation results
|   `-- faithfulness_eval_*/      # Timestamped evaluation results
|       `-- reports/              # Generated reports
|           `-- report_*/         # Timestamped report directory
|               |-- final_metrics_{model_name}.json    
|               |-- type_metrics_{model_name}.json     
|               |-- sample_results_{model_name}.json   
|               `-- visualizations/                    
|-- visualizations/              # Model comparison charts
|   |-- model_comparison.png    # Model performance comparison
|   `-- type_comparison.png     # Sample type performance comparison
`-- environment.yml             # Project dependencies
\end{lstlisting}

\vspace{0.5em}
\subsection{Core Components Analysis}

\subsubsection{Evaluation Core (eval.py)}
The \texttt{FaithfulnessEval} class implements the core evaluation process:

\begin{lstlisting}[
    language=Python,
    breaklines=true,
    basicstyle=\ttfamily\footnotesize,
    commentstyle=\color{gray}
]
class FaithfulnessEval(Eval):
    # Define evaluation metrics
    METRICS = [
        "factual_accuracy",
        "logical_coherence",
        "context_relevance",
        "interpretative_reasoning",
        "information_completeness",
        "hallucination_score"
    ]
    
    # Define evaluation weights for different types
    TYPE_WEIGHTS = {
        "general": {
            "factual_accuracy": 0.30,
            "logical_coherence": 0.20,
            "context_relevance": 0.15,
            "interpretative_reasoning": 0.15,
            "information_completeness": 0.10,
            "hallucination_score": 0.10
        }
    }
    
    def eval_sample(self, sample: Dict[str, Any]) -> Dict[str, Any]:
        # Get key information from sample
        context = sample.get("context", "")
        query = sample.get("query", "")
        reference = sample.get("reference", "")
        sample_type = sample.get("type", "general")
        
        # Calculate metrics
        metrics = {
            "factual_accuracy": self.metrics.calculate_factual_accuracy(
                response, reference),
            "logical_coherence": self.metrics.calculate_logical_coherence(
                response),
            "context_relevance": self.metrics.calculate_context_relevance(
                response, context)
        }
        
        return {
            "sample": sample,
            "response": response,
            "metrics": metrics,
            "type": sample_type
        }
\end{lstlisting}

\vspace{0.5em}
Key features include:
\begin{itemize}
    \item Sample loading and preprocessing
    \item Model response acquisition
    \item Metric calculation and scoring
    \item Dynamic weight adjustment
    \item Result aggregation and reporting
\end{itemize}

\subsubsection{Metrics Implementation (metrics.py)}
The \texttt{FaithfulnessMetrics} class provides metric calculations:

\begin{lstlisting}[language=Python, breaklines=true, basicstyle=\ttfamily\footnotesize]
class FaithfulnessMetrics:
    def calculate_context_relevance(self, response: str, context: str) -> float:
        # Calculate semantic relevance
        semantic_relevance = self.calculate_semantic_similarity(
            response, context)
        
        # Calculate information coverage
        coverage_score = self.calculate_coverage(response, context)
        
        # Calculate topic consistency
        topic_score = self.calculate_topic_consistency(response, context)
        
        # Final weighted score
        final_score = (
            semantic_relevance * 0.4 +
            coverage_score * 0.3 +
            topic_score * 0.3
        )
        
        return float(final_score)
        
    def calculate_overall_faithfulness(self, metrics: Dict[str, float]) -> float:
        # Base weights
        base_weights = {
            "factual_accuracy": 0.25,
            "logical_coherence": 0.15,
            "context_relevance": 0.15,
            "interpretative_reasoning": 0.15,
            "information_completeness": 0.15,
            "hallucination_score": 0.15
        }
        
        # Dynamic weight adjustment
        if metrics.get("factual_accuracy", 0) < 0.5:
            base_weights["factual_accuracy"] = 0.35
            base_weights["hallucination_score"] = 0.20
            # Adjust other weights...
        
        return float(sum(metrics[m] * w for m, w in base_weights.items()))
\end{lstlisting}

\subsubsection{Report Generation (report.py)}
The \texttt{FaithfulnessReport} class handles report generation:

\begin{lstlisting}[language=Python, breaklines=true, basicstyle=\ttfamily\footnotesize]
class FaithfulnessReport:
    def generate_report(self, final_metrics, type_metrics, sample_results):
        # Generate visualizations
        self._generate_radar_charts()
        self._generate_heatmaps()
        self._generate_boxplots()
        self._generate_trend_plots()
        
        # Generate report content
        content = []
        content.append("# Evaluation Report\n")
        
        # Add overall metrics
        content.append("## 1. Overall Metrics")
        for metric, score in final_metrics.items():
            content.append(f"- {metric}: {score:.4f}")
        
        # Add type-specific results
        content.append("\n## 2. Type-Specific Results")
        for type_name, metrics in type_metrics.items():
            content.append(f"\n### {type_name}")
            for metric, score in metrics.items():
                content.append(f"- {metric}: {score:.4f}")
        
        return "\n".join(content)
\end{lstlisting}

\subsection{Environment Setup}

\subsubsection{Required Tools and Dependencies}
Essential components:
\begin{itemize}
    \item Python 3.9 or higher
    \item Anaconda/Miniconda for environment management
    \item OpenAI API access credentials
    \item Required Python packages:
    \begin{itemize}
        \item transformers
        \item torch
        \item nltk
        \item openai
        \item numpy
        \item pandas
    \end{itemize}
\end{itemize}

\subsubsection{Installation Process}
Setup steps:
\begin{enumerate}
    \item \textbf{Environment Creation}:
    \begin{lstlisting}[breaklines=true, basicstyle=\ttfamily\footnotesize]
conda env create -n evals -f environment.yml
conda activate evals
    \end{lstlisting}

    \item \textbf{Package Installation}:
    \begin{lstlisting}[breaklines=true, basicstyle=\ttfamily\footnotesize]
pip install -r requirements.txt
    \end{lstlisting}

    \item \textbf{NLTK Resources}:
    \begin{lstlisting}[breaklines=true, basicstyle=\ttfamily\footnotesize]
python -m nltk.downloader punkt wordnet
    \end{lstlisting}

    \item \textbf{API Configuration}:
    \begin{lstlisting}[breaklines=true, basicstyle=\ttfamily\footnotesize]
export OPENAI_API_KEY="your-api-key"
    \end{lstlisting}
\end{enumerate}

\subsection{Samples and Configuration}

\subsubsection{Evaluation Samples}
Sample types in \texttt{samples.jsonl}:
\begin{itemize}
    \item \textbf{Scientific Explanations}:
    \begin{itemize}
        \item Complex scientific concepts
        \item Technical processes
        \item Research findings
    \end{itemize}
    \item \textbf{Technical Analyses}:
    \begin{itemize}
        \item System architectures
        \item Algorithm explanations
        \item Performance evaluations
    \end{itemize}
    \item \textbf{Medical Advice}:
    \begin{itemize}
        \item Health recommendations
        \item Treatment explanations
        \item Medical procedures
    \end{itemize}
    \item \textbf{Historical Analyses}:
    \begin{itemize}
        \item Historical events
        \item Cultural developments
        \item Societal changes
    \end{itemize}
    \item \textbf{Current Events}:
    \begin{itemize}
        \item Recent developments
        \item Ongoing situations
        \item Contemporary issues
    \end{itemize}
\end{itemize}

\subsubsection{Framework Configuration}
Settings in \texttt{faithfulness.yaml}:
\begin{itemize}
    \item \textbf{Evaluation Parameters}:
    \begin{itemize}
        \item Metric weights and thresholds
        \item Model configuration settings
        \item Evaluation criteria
    \end{itemize}
    \item \textbf{Report Settings}:
    \begin{itemize}
        \item Output format preferences
        \item Visualization options
        \item Data export configurations
    \end{itemize}
\end{itemize}

\subsection{Usage Example}
Basic evaluation workflow:

\begin{lstlisting}[language=Python, breaklines=true, basicstyle=\ttfamily\footnotesize]
from evals.elsuite.faithfulness.eval import FaithfulnessEval
from evals.record import RecorderBase
from evals.completion_fns.openai import OpenAIChatCompletionFn

# Create completion function
completion_fn = OpenAIChatCompletionFn(model="gpt-3.5-turbo")

# Create evaluator instance
evaluator = FaithfulnessEval(
    completion_fns=[completion_fn],
    eval_registry_path="evals/registry/evals/faithfulness.yaml",
    samples_jsonl="evals/registry/data/faithfulness/samples.jsonl",
    report_dir="reports"
)

# Create recorder
recorder = RecorderBase()

# Run evaluation
results = evaluator.run(recorder)
print(f"Overall Faithfulness Score: {results['overall_faithfulness']:.4f}")
\end{lstlisting}